% Options for packages loaded elsewhere
\PassOptionsToPackage{unicode}{hyperref}
\PassOptionsToPackage{hyphens}{url}
%
\documentclass[
]{article}
\usepackage{lmodern}
\usepackage{amssymb,amsmath}
\usepackage{ifxetex,ifluatex}
\ifnum 0\ifxetex 1\fi\ifluatex 1\fi=0 % if pdftex
  \usepackage[T1]{fontenc}
  \usepackage[utf8]{inputenc}
  \usepackage{textcomp} % provide euro and other symbols
\else % if luatex or xetex
  \usepackage{unicode-math}
  \defaultfontfeatures{Scale=MatchLowercase}
  \defaultfontfeatures[\rmfamily]{Ligatures=TeX,Scale=1}
\fi
% Use upquote if available, for straight quotes in verbatim environments
\IfFileExists{upquote.sty}{\usepackage{upquote}}{}
\IfFileExists{microtype.sty}{% use microtype if available
  \usepackage[]{microtype}
  \UseMicrotypeSet[protrusion]{basicmath} % disable protrusion for tt fonts
}{}
\makeatletter
\@ifundefined{KOMAClassName}{% if non-KOMA class
  \IfFileExists{parskip.sty}{%
    \usepackage{parskip}
  }{% else
    \setlength{\parindent}{0pt}
    \setlength{\parskip}{6pt plus 2pt minus 1pt}}
}{% if KOMA class
  \KOMAoptions{parskip=half}}
\makeatother
\usepackage{xcolor}
\IfFileExists{xurl.sty}{\usepackage{xurl}}{} % add URL line breaks if available
\IfFileExists{bookmark.sty}{\usepackage{bookmark}}{\usepackage{hyperref}}
\hypersetup{
  pdftitle={无穷小量的前世今生.md},
  hidelinks,
  pdfcreator={LaTeX via pandoc}}
\urlstyle{same} % disable monospaced font for URLs
\setlength{\emergencystretch}{3em} % prevent overfull lines
\providecommand{\tightlist}{%
  \setlength{\itemsep}{0pt}\setlength{\parskip}{0pt}}
\setcounter{secnumdepth}{-\maxdimen} % remove section numbering

\title{无穷小量的前世今生.md}
\author{}
\date{}

\begin{document}
\maketitle

\hypertarget{header-n0}{%
\section{无穷小量的前世今生}\label{header-n0}}

\hypertarget{header-n2}{%
\paragraph{刘正浩 2019270103005}\label{header-n2}}

\hypertarget{header-n3}{%
\subsubsection{0. 前奏}\label{header-n3}}

无穷小量,一个听起来既熟悉又陌生的名词。我们在学习数学的过程中一定都听说过``无穷小量''这个概念,但无穷小量究竟是什么?打个比方好了:如果说数学是人类智力的那一顶王冠,那么``无穷小量''可以说是这顶王冠上最为璀璨的宝石之一。

从古希腊到工业革命,从东方到西方,人们对无穷小量的追寻从未停止过。无数哲学家、数学家的思考与探究,让这个如幽灵一般诡秘的概念逐渐变得清晰起来;我们也能从中领悟到,数学,这座人类文明史上最宏伟壮丽的大厦,是怎样耗尽先贤的心血、聚集先贤的才智,一砖一瓦地被建立起来的。

\hypertarget{header-n6}{%
\subsubsection{1. 隐匿的幽灵}\label{header-n6}}

早在古希腊,哲学家芝诺就提出了一个著名的佯谬:

\emph{一只乌龟和人赛跑,人的速度比乌龟快一百倍,但在起跑时乌龟领先人99个单位。那么人能否追上乌龟?}

现在我们都知道,人一定会追上乌龟,而且我们很轻易地就可以算出人追上乌龟花费的时间:

\[\frac{{99}}{{100 - 1}} = 1个单位时间\]

然而,芝诺却不这么认为。他是这样分析的:人要想追上乌龟,必须比乌龟多走过
\(99\) 个单位;但在人走过 \(99\) 个单位的同时,乌龟又向前走了 \(0.99\)
个单位;人在追上这 \(0.99\) 个单位的同时,乌龟又向前走了 \(0.0099\)
个单位\ldots\ldots 也就是说,在芝诺的眼中,人永远追不上乌龟。

换句话说,芝诺认为\textbf{人与乌龟之间很小的距离(也就是无穷小量)是永远不可能被忽略掉的,是永远不可能等于
\(0\)
的}。这在``正常人''的眼中是多么不合理的结论!由此可以看出,古希腊人已经拥有了关于无穷小量的模糊概念,但正因为芝诺佯谬的存在,古希腊的几何证明(当时的``数学'')中就此开除了无穷小量的``数学籍''。

时间来到芝诺悖论之后七百年。在遥远的东方,数学家刘徽提出了计算圆周率的著名方法------割圆术。

在割圆术中,他用弦代替弧计算多边形的面积来近似计算圆的面积,并在《九章算术注》中这样描述他的割圆术:\emph{``割之弥细,所失弥少,割之又割,以至于不可割,则与圆合体,而无所失矣。''}翻译成白话就是:(多边形)分割得越细,多边形与圆的差距就越来越小,一直分割一直分割,到了不能再继续分割的时候,多边形就和圆重合了,它们的面积差也就消失了。

很明显,刘徽认为,\textbf{用来近似圆的多边形与圆之间面积之差(在分割数很大时同样是一个无穷小量),随着分割数越来越大而变得可以被忽略掉,也就是无穷小量可以等于
\(0\) !}这和芝诺得到的结论刚好相反!

所以,无穷小量究竟能不能被忽略掉?它究竟是什么?是不是 \(0\)
?抑或是只有在某些情况下它才能被看做
\(0\)?在古代哲学家、数学家对数学的讨论过程中,就这样埋下了一颗名为无穷小量的定时炸弹。这个如幽灵般徘徊在数学世界上空的模糊概念,将在将来爆发出来,引发著名的第二次数学危机。

\hypertarget{header-n17}{%
\subsubsection{2. 幽灵现身}\label{header-n17}}

由于古希腊的数学家们不把无穷小量作为``数学''中的一个概念,自第一次数学危机(希帕索斯发现无理数)之后的几百年间,数学这只``怪兽''始终以温和而又完美的姿态面对着人类------在人们看来,数学即将成为一个完美的体系,这个体系中没有任何的疑点,它就是真理的化身,可以用来解释世间万物。

可真理哪有这么容易就能来到人们身边呢?既然无穷小量是真实存在的,那么将它排除在外的``数学''又怎么可能是完美的体系?在以往的探究中埋下的这颗定时炸弹,即将爆发。

在公元十七世纪,两位伟人------牛顿和莱布尼茨诞生了。现在我们都说,是他们分别独自创立了微积分这一门学科。但实际上,在他们之前,还有许多伟人也对微积分这一学科有过贡献:

\begin{itemize}
\item
  开普勒给出了公式:
\end{itemize}

\[\int_0^\theta  {\sin \theta d\theta  = } 1 - \cos \theta \\]

\begin{itemize}
\item
  卡瓦列里给出了公式:
\end{itemize}

\[\int_0^x {{t^n}dt = } \frac{{{x^{n + 1}}}}{{n + 1}}\\]

\begin{itemize}
\item
  费马给出了导数思想的雏形,以及著名的费马引理:若函数
  \(f:(a,b)\rightarrow R\) 在 \(x_0∈(a,b)\) 处取得极值,且 \(f\) 在
  \(x_0\) 处可导,则 \(f'(x_0)=0\) ;
\item
  帕斯卡注意到,很小的弧与切线可以互相替代,并在证明体积公式时略去了高阶无穷小;
\item
  巴罗(牛顿的恩师)给出了求切线的方法。
\end{itemize}

但不要误会,他们的``微积分''与我们现在所熟知的微积分有很大的差别。差别在哪里?差别正在于他们的``微积分''中的\textbf{极限理论是不完整的,使用的``无穷小量''这一概念更是扑朔迷离}。

那么话说回来,无穷小量究竟是怎样引发一场数学危机的?

举一个例子好了:求 \(f(x)=x^2\) 在 \(x=2\) 处的导数。牛顿是这样计算的:

\[f'(2) = \frac{{{{(2 + \alpha )}^2} - {2^2}}}{\alpha } = \frac{{4\alpha  + {\alpha ^2}}}{\alpha } = 4 + \alpha  = 4\\]

牛顿把 \(\alpha\)
叫做``无穷小量''。猛一看这个式子好像像模像样的没什么毛病,于是牛顿就由此把他的理论推广开来,并用它解决了大量以前无法解决的问题。

然而,当时英国的大主教贝克莱发现了这其中的矛盾。贝克莱于1734年专门写了一本书,攻击流数(导数)\emph{``是消失了的量的鬼魂\ldots\ldots 能消化得了二阶、三阶流数的人,是不会因吞食了神学论点就呕吐的。''}他说,用忽略高阶无穷小来消除了原有错误,\emph{``是依靠双重的错误得到了虽然不科学却是正确的结果''}。他是这样反驳牛顿的:要使前两个式子
\(\frac{{{{(2 + \alpha )}^2} - {2^2}}}{\alpha }\) 、
\(\frac{{4\alpha  + {\alpha ^2}}}{\alpha }\) 成立,\(\alpha\) 必须不等于
\(0\);但到了第三个式子 \(4 + \alpha\) 中,\(\alpha\) 居然又等于 \(0\)
了\ldots\ldots 他还讽刺挖苦牛顿说:\emph{既然分子和分母都变成``无穷小''了,而无穷小作为一个量,\textbf{既不是
\(0\) ,又不是非 \(0\) },那它一定是``量的鬼魂''了!}

虽然贝克莱提出疑问的动机并不是促进数学的发展,但他提出的这个问题对新兴的微积分学来说却是致命的:它动摇了微积分学的根基。

在此后的两百年间,有无数数学家前赴后拥,尝试对无穷小量悖论展开证明。例如:麦克劳林试图从瞬时速度的方面对无穷小量进行阐述,泰勒则试图用差分法解释无穷小量,等等。但可惜的是,他们都没有成功,数学家们始终没能摆脱``无穷小量''这个飘荡的幽灵。

\hypertarget{header-n44}{%
\subsubsection{3. 幽灵的消散}\label{header-n44}}

之所以会有关于无穷小量的悖论的出现,还是由于极限理论是不完善的,它还存在很大的缺陷。没有对极限这一概念的完美的阐述,也就不可能清晰地描述无穷小量这个概念。此时的微积分学就像一栋地基不牢的大厦,看上去壮丽辉煌、夺人眼球,但实际上马上就要倾倒,让几代人的心血付诸东流。

好在江山代代出人才,在贝克莱提出质疑的两百年后,逐渐有数学家醒悟过来,开始对极限这一概念进行详细严谨的阐述。

数学家达朗贝尔首先将微积分研究的问题转化成极限的问题,并这样解释极限:\emph{``一个变量趋近于一个固定的的量,并且趋近的程度小于任何给定的量''}。在这个阐述中,已经能看到
\(\varepsilon  - \delta \)
语言的影子了。然而,达朗贝尔同意逐渐消失的量(即无穷小量)是没有意义的。他斩钉截铁般地宣称:\emph{``一个量要么是有,要么是没有。如果是有,它就不可能消失;如果是没有,它就确实消失了。假设存在介于这两者之间的中间状态,那它就只能是一头由狮头羊身和蛇尾构成的吐火怪物。''}现在我们知道,他关于极限的表述有部分是正确的,但他关于无穷小量的认识却是错误的:无穷小量当然真实存在!

之后,柯西在1821年出版的《分析教程》中如此给出极限的定义:\emph{``当属于一个变量的相继值无限地趋近某个固定值时,如果以这样一种方式告终,变量值同固定值之差小到我们希望的\textbf{任意小},那么这个固定值就称为其他所有值的极限''}。柯西认为,极限是一个无限趋近的过程,是动态的、变化的。除此之外,他还给出了无穷小的定义:\emph{``当一个变量的连续数值无限减小(从而变得小于任何给定的值)时,这个变量就称为一个无穷小的量''}。但很可惜的是,柯西对``无穷小量的值究竟是多少''这个问题闭口不谈,他仍然没有把无穷小量这个概念解释清楚。

后来,经过魏尔斯特拉斯、戴德金、康托尔等数学家的不懈努力,数学家们通过把极限论建立在严格的实数理论基础上的方法,形成了严谨地定义极限的
\textbf{\(\varepsilon  - \delta \) 语言}。

例如,用 \(\varepsilon  - \delta \) 语言描述 \(x \to x_0\)
时的函数极限:

设 \(f:\mathop U\limits^0 ({x_0}) \to R\) 是任一函数,若存在常数
\(a \in R\) ,它与 \(f(x)\)
满足如下关系:\(\forall \varepsilon  > 0\),\(\exists \delta  > 0\),使得当
\(0 < \left| {x - {x_0}} \right| < \delta \) 时,恒有
\(\left| {f(x) - a} \right| < \varepsilon \) ,则称 \(\alpha\) 是当
\(x \to x_0\) 时 \(f(x)\) 的极限。

有了 \(\varepsilon  - \delta \)
语言,我们就可以轻易地完美解释贝克莱提出的疑问:对于 \(f(x)=x^2\) 在
\(x=2\) 处的导数,

\[f'(2) = \mathop {\lim }\limits_{\Delta x \to 0} \frac{{{{(2 + \Delta x)}^2} - {2^2}}}{{\Delta x}} = \mathop {\lim }\limits_{\Delta x \to 0} \frac{{4\Delta x + {{(\Delta x)}^2}}}{{\Delta x}} = \mathop {\lim }\limits_{\Delta x \to 0} (4 + \Delta x) = 4\]

从上面的式子就可以看出,用极限来描述求导的过程,比起牛顿那种粗糙的做法是多么地优雅而又严谨!

这样一来,无穷小量也就可以借助极限理论来进行明确的定义了:

当 \(x \to x_0\)(或 \(x \to \infty\))时,以零为极限的\textbf{函数}
\(a(x)\) 称为当 \(x \to x_0\)(或 \(x \to \infty\))的无穷小量。

\textbf{无穷小量这个幽灵就此烟消云散}。我们可以清楚地知道,无穷小量是一个以
\(0\) 为极限的函数,它不是 \(0\)
,在计算中也绝对不能把它与绝对值很小的变量混为一谈。并且,一个函数是否是无穷小,与自变量的变化趋势也息息相关。例如:\(\frac{1}{x}\)
是 \(x \to \infty\) 时的无穷小,因为只有当 \(x \to \infty\) 时才有
\(\frac{1}{x} \to 0\) ;当 \(x \to x_0\) 时,\(\frac{1}{x}\)
趋于一个有限值 \(\frac{1}{x_0}\) ;当 \(x \to 0\) 时,
\(\frac{1}{x} \to \infty\)。在后两种情况下,\(\frac{1}{x}\)
显然不是无穷小。

既然微积分学最基础的理论------极限理论已经被精准地表述出来,余下内容的发展与完善也就是水到渠成的了。至于微积分学之后发展的详细过程,这里按下不表。

\hypertarget{header-n59}{%
\subsubsection{4. 终章}\label{header-n59}}

人们对无穷小量的几千年探索历史告诉我们,任何一门学科的建立容不得半点马虎,所有的定义都必须是准确的、精炼的,否则这个学科即使发展的再快,内容再丰富,也会因为基础不牢而被驳倒甚至全盘否定。

同时,我们也可以看到,任何学科的发展都不是一步到位的,而是经过无数次``发现错误→修正→发现新的错误→再修正''的过程才逐渐完善起来。

我个人认为胡适说得很对:``大胆假设,小心求证''。科学研究者应当具有这样的特性:在对未知领域进行假设时一定要足够大胆,因为在研究前完全无法知晓未知领域究竟是什么样子的;但在验证假设时一定要小心小心再小心,因为在验证的过程中稍有纰漏就会使结论与事实相差甚远,正所谓``失之毫厘,谬以千里''是也。

\end{document}
