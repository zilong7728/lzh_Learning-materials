\documentclass[a4paper]{article}
\usepackage[margin=1in]{geometry}%设置边距,符合Word设定
\usepackage{amssymb,amsfonts,amsmath,amsthm}
\usepackage{ctex}
\usepackage{setspace}
\usepackage{lipsum}
\usepackage{graphicx}%插入图片
\graphicspath{{Figures/}}%文章所用图片在当前目录下的 Figures目录

\usepackage{hyperref} % 对目录生成链接,注:该宏包可能与其他宏包冲突,故放在所有引用的宏包之后
\hypersetup{colorlinks = true,  % 将链接文字带颜色
	bookmarksopen = true, % 展开书签
	bookmarksnumbered = true, % 书签带章节编号
	pdftitle = 电磁波扩展作业, % 标题
	pdfauthor =刘正浩 2019270103005} % 作者
\bibliographystyle{plain}% 参考文献引用格式
\newcommand{\upcite}[1]{\textsuperscript{\cite{#1}}}

\renewcommand{\contentsname}{\centerline{Contents}} %经过设置word格式后,将目录标题居中


\title{\heiti\zihao{2} 电磁波扩展作业}				%title
\author{\songti 刘正浩 2019270103005}
\date{\today}


\begin{document}
	\maketitle
	\thispagestyle{empty}

%\begin{abstract}
%	\lipsum[2]
%\end{abstract}

	\tableofcontents

	\section{第一题}
		由于在平行(垂直)极化波中的振动方向只有一个方向的单位向量,
		所以不可能出现垂直于这个方向的单位向量,也就不可能出现任何含有垂直的单位向量成分的极化方向

	\section{第二题}
		由于反射系数与极化向量的乘积得到反射波的极化向量,根据计算可知反射波的两个相位之差仍然与入射波的两个相位之差,
		所以反射波的旋向与入射波的旋向一定相等;但由于反射波的传播方向与入射波相反,所以综合起来看极化方向是翻转的。
	\end{document}